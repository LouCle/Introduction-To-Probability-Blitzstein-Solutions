\begin{enumerate}[label=(\alph*)]
\item From the student's perspective, the average class size is $\text{E}(X)
=
\frac{200}{360}100 + \frac{160}{360}10 = 60$. From the dean's perspective, the
average class size is $\text{E}(X) = \frac{16}{18}10 + \frac{2}{18}100 =
20$. The discrepancy comes from the fact that when surveying the dean, there are
only two data points with a large number of students. However, when surveying
students, there are two hundred data points with a large number of students.
In a sense, the student's perspective overcounts the classes.

\item Let $C$ be a set of $n$ classes with $c_{i}$ students for $1 \leq i \leq
n$. The dean's view of average class size then is $\text{E}(X) = \sum_{i=1}^
{n}\frac{c_{i}}{n}$. The students' view of average class size is $\text{E}
(X) = \sum_{i=1}^{n}(c_{i}\frac{c_{i}}{\sum_{i=1}^{n}c_{i}})$. In the dean's
perspective, all $c_{i}$ are equally weighted - $\frac{1}{n}$. However, in the
students' perspective, weights scale with the size of the class. Thus, the
students' perspective will always be larger than the dean's, unless all classes
have the same number of students.
\end{enumerate}