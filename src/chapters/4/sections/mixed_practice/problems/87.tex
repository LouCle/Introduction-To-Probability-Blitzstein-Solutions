\begin{enumerate}[label=(\alph*)]
\item Let $I_{j}$ be the indicator random variable for $j$ person in the sample
being a democrat. Let $X$ be the total number of democrats in the sample. Then,
$\text{E}(X) = \sum_{j=1}^{c}I_{j} = c\frac{d}{100}$

\item Let $I_{j}$ be the indicator random variable for state $j$ being
represented by at least one person in the sample. Then $P(I_{j} = 1) = 
1 - \frac{\binom{98}{c}}{\binom{100}{c}}$. Then, the expected number of states
represented in the sample is $\text{E}(X) = 50(1 - \frac{\binom{98}{c}}{
\binom{100}{c}})$.

\item Similarly to part $b$, $\text{E}(X) = 50\frac{\binom{98}{c-2}}{\binom{100}
{c}}$.

\item $P(X = k) = \frac{\binom{50}{k}\binom{50}{20-k}}{\binom{100}{20}}$ for $0
\leq k \leq 20$.

Letting $I_{j}$ be the indicator variable for person $j$ in the sample being a
junior senator of a state, $\text{E}(X) = 20\frac{50}{100} = 10$.

\item Similar to part $b$, $\text{E}(X) = 50\frac{\binom{98}{18}}{\binom{100}
{20}}$.

\end{enumerate}